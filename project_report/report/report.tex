\documentclass{acm_proc_article-sp}

\usepackage{natbib}
\usepackage{url}

\makeatletter
\def\@copyrightspace{\relax}
\makeatother

\begin{document}

\title{SMS Immunization Manager (SIM)}

\numberofauthors{4} 
\author{
       % \alignauthor
       Jenny Kang, Isaac Reynolds, Jackson Roberts, Nicholas Shahan\\
              \affaddr{University of Washington}\\
              \affaddr{Seattle, WA, USA}\\
              \email{jskang, isaacr, jcwr, nshahan@uw.edu}
}

\maketitle
\begin{abstract}
Short description of your project, what you accomplished, and what conclusions 
you were able to draw. This should be in the range of 250-300 words.
\end{abstract}

\section{Introduction}
A coordinated, national effort (supported by organizations such as UNICEF, WHO, and thousands of NGOs and universities) to distribute vaccines in a developing country has proven to be among ``the most successful and cost-effective health interventions'' in human history. These programs currently save between two and million deaths per year. However, even as recently as in 2008, nearly 17\% of the 8.8 million deaths in infants and young children were from diseases that could have been prevented by vaccines \cite{who:campaign_essentials}. 

Improving the effectivness of vaccine distribution programs requires improving networks in existing deployment programs as well as instituting new programs in poorly-covered countries. Our project focuses on improving an existing deployment system in Laos, which will be discussed below. Many of the processes and motivations described below are similar for other countries as well.

The current system in Laos distributes vaccines down a hierarchy on a monthly cycle. Each step lower in the hierarchy contains smaller facilities with progressively less lower capacity, fewer staff, less frequent restocking, poorer roads, and less reliable equipment, power, cell service, and Internet. Each vaccine must be stored at a specific, cool temperature to avoid spoiling, so this chain of facilities through which vaccines pass is called the ``cold chain''.

The cold chain at the lowest tier of the hierarchy operates on roughly a monthly schedule (sometimes less for particularly remote stations). Every month, workers at the next higher tier fill a small truck with ice chests full of vaccine (once at an outputs, the vaccines are kept in propane- or electicity-powered refrigerators) and then spend about two weeks driving a loop through these outputs. At each facility, the workers collect information about the previous month's demand, any necessary repairs, and the amount of vaccine spoilage. This information is used to plan the next deliveries. 

Most vaccinations are done by the Community Health Workers at these small facilities. People travel for hours by foot to get their children vaccinated, often forgoing other vital tasks. They either travel directly to the outpost, or to an event in some other community that a CHW has scheduled in advance. Sometimes CHWs pack bags with vaccines and walk around neighboring communities to do vaccinations. 

Being vaccinated is a large time investment for most individuals, so it's important that there's enough unspoiled stock to satisfy the demand for vaccinations. Individuals who travel for hours only to find that there is no vaccine left are less likely to return again, and they may recommend the same to others. 

Running out of stock is a real concern. High demand, low supply, a broken or near-broken fridge, interruptions in power supply, or temperatures high or low enough to overwhelm fridges can all cause unexpected stock outages. Giving national-, provincial-, and district-level distributors timely access to reliable information can prevent stock outages as well as mitigate the effects of outages that do occur. 

Unfortunately, the current data collection schedule means that each facility is resupplied based on data that is, on average, six weeks old. Any improvement in this aspect of distribution requires an alternate form of reporting. 

\subsection{SMS-Based Reporting}

We present SMS Immunization Manager (SIM), a framework that allows CHWs to report information about the vaccine cold chain (such as fridge temperature and condition) quickly by SMS. SIM's main features include 

\begin{itemize}
\item Robust SMS-based operations that correctly handle even poorly-formatted messages through careful design of message formats. The included operatations were inspired by a current need in Laos.
\item A moderation web application for administrators to review the system's operation.
\item A simple and well-documented design that enables one or two developers to deploy a complete system in only several weeks.
\end{itemize}

The system will be deployed in Laos in mid-2014, but it is useful for any cold chain reporting system and could easily be adapted for new deployments. A diagram describing how SIM interacts with surrounding systems is shown below.

\begin{figure*}
\centering
\epsfig{file=sim-broad-diagram.pdf}
\caption{SIM's relationship with other systems.}
\end{figure*}

SIM communicates exclusively via HTTP, which means it can be deployed on any web host. In order for CHWs to communicate with SIM via SMS, a ``gateway device'' must be deployed on the CHW's cellular network or a connected cellular network. The gateway device may be as simple as a midrange Android phone (the system was originally implemented using EnvayaSMS running on such a phone), or it may be as complex as a high-load custom-built gateway provided by the cellular provider. 

Although SIM was built to handle poorly-formatted SMS messages typed by hand by people with little access to training, there is no requirement that messages are hand-typed. The messages could just as easily be created by a simple application running on an embedded controller --- for example, the FoneAstra system could automate data reporting by sending properly-formed SMS messages to SIM \cite{foneastra}.

The choice of where and when to use SMS and HTTP protocols was made carefully. The majority of users, especially CHWs reporting cold chain data, are without reliable access to Internet and power. This required limiting the use of HTTP. However, cell coverage in developing countries is good; recent estimates say that roughly five billion people in developing countries have access to a cell phone \cite{worldbank:mobileaccess}. SIM only requires several minutes of coverage per report and expects reports one or two times per month.

However, SIM requires the gateway to have Internet access so that it may relay messages between SMS users and the SIM server. It also requires that moderators have Internet access. This allows SIM to provide a powerful, usable, bug-free moderation interface to review the system's operation.

In order to prevent malicious or accidental misuse, SIM only allows access from registered users (SMS users and moderators), each of whom is assigned one of several ``roles'' (with associated permissions). Among other things, privileged SMS users may register new users via SMS (for example, the facility manager would be able to register the other workers), and privileged moderators can access special tools for administering the Django and RapidSMS architectures on which SIM is based. This user information, which also includes each user's preferred language, is kept in a database managed by SIM.

However, SIM does not attempt to store any more cold chain information than necessary (typically only the facility hierarchy). Instead, SIM is intended to pass all of this data to a complete remote cold chain information/equipment management system such as DHIS2. That remote database will become the permanent repository for cold chain data.

CHWs interact with SIM by sending structured text messages to SIM. Common use cases in Laos include

\begin{itemize}
\item A CHW periodically (monthly or more) sends one SMS message including both (a) their facility's current stock of each vaccine and (b) the number of times each of their fridges dipped below or rose above ideal temperature during the reporting period. Fridge data in Laos is recorded automatically by a Fridge-tag \textsuperscript{\textregistered} \cite{fridgetag}. If the CHW's message is too poorly-formatted to interpret, within several minutes a useful error message is returned (otherwise, a ``thank you'' message is returned).
\item A facility unexpectedly runs out of stock of a particular vaccine, so a worker at the facility sends, as soon as possible, an SMS message identifying the vaccine. The stock outage is brought to the attention of the moderator the next time they log in so the outage can be handled rapidly. The same process can be used if any equipment fails.
\item It is a fact of life that workers come and go, change facilities, and change SIM cards. Privileged SMS users can register a new data reporter with a name, phone number, and associated facility by sending a single SMS message. Any worker can also change their preferred language at any time and see all SMS responses in that language.
\end{itemize}

Moderators interact with SIM by using a web application. Common use cases include

\begin{itemize}
\item Any incoming SMS messages that failed (for syntax or permissions reasons) are brought to the attention of a moderator. The notification persists in the moderation application until the moderator explicitly views and dismisses the source of the error. Errors are viewing in the context of any of that sender's other messages, which allows moderators to track individuals and make informed decisions about the effectiveness of the system.
\item When viewing an SMS message with an error, a moderator can edit the message and resend it (internally, without incurring SMS charges) with any explicit permissions level. This gives moderators the power to quickly correct messages that the system cannot understand but a human can.
\item Moderators can view per-message logs of every side effect the message produced and each operation the system applied to the message (including parsing, verifying permissions, filtering for spam, etc.). This allows moderators to quickly identify bugs in the system and provide accurate and useful bug reports to developers.
\end{itemize}

The project is driven by goals to 

\begin{itemize}
\item Define unambiguous syntaxes so that poorly-formatted messages may be parsed successfully and no delimiters are necessary.
\item Provide an architecture in which new SMS operations may be easily implemented and integrated into the moderation interface and the side effects of existing or new SMS operations are easy to add, remove, and modify.
\item Provide extensive documentation so that a complete, reliable solution can be implemented and deployed by a team of one or two developers in just several weeks.
\end{itemize}

\section{Related Work}
What have others done - both closely and loosely related?  It is important to make sure you do a good job of this and draw upon a wide range of information sources including conferences and journals in the area, magazines and popular press, newsletters, and, of course, web searches.  Feel free to build on what was done in Winter quarter but go further.  In particular make sure you do a survey of related technologies, not just research projects.  This is where the bulk of the citations will be and should be about 1 page clearly explaining why these other works were not enough to solve the problem.

\section{Approach}
How are you going about realizing your idea?  What are the main pieces?  Why did you choose these pieces?  How do they interact?  How generally reusable will these pieces be?  This should be a fairly large section with several figures and should probably run about 1.5 to 2 pages.

\section{Implementation}
Describe the choices you had to make that were not obvious.  What was hard about getting things to work?  What tradeoffs did you have to consider?  You also want to talk about how things could have been implemented better if you had more resources or more time.  This section will also be about 1.5 to 2 pages and should include figures about the data flow and final design.

\section{Evaluation}
How are you evaluating your solution?  Evaluation has both technical aspects as well as usability aspects.  Who are your target users and who did you use as surrogates?  How well does it work?  Use both qualitative and quantitative approaches and relate the evaluation back to the scenario.  Another 1.5 to 2 page section.

\section{Conclusion and Future Work}
What are the next steps to take and why?  What lessons did you learn?  What would you differently next time?  How could the solution be improved upon further?  This section can be short and should recap the paper and back up the claims made in the abstract.  It is only about a page in length.

\section{Acknowledgements}
Who helped you along the way this quarter and how?

\bibliographystyle{abbrv}
\bibliography{report}  
\balancecolumns

\appendix
All the material you might need to reconstruct your project: source code, schematics, installation and configuration instructions, etc.  (NOTE: this does not have to conform to the template provided and can be in the most appropriate format to make it easiest to use).

\end{document}
