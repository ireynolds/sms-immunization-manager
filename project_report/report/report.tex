\documentclass{acm_proc_article-sp}

\usepackage{natbib}
\usepackage{url}

\begin{document}

\title{SMS Immunization Manager (SIM)}

\numberofauthors{4} 
\author{
       % \alignauthor
       Jenny Kang, Isaac Reynolds, Jackson Roberts, Nicholas Shahan\\
              \affaddr{University of Washington}\\
              \affaddr{Seattle, WA, USA}\\
              \email{jskang, isaacr, jcwr, nshahan@uw.edu}
}

\maketitle
\begin{abstract}
Short description of your project, what you accomplished, and what conclusions 
you were able to draw. This should be in the range of 250-300 words.
\end{abstract}

\section{Introduction}
A coordinated, national effort (supported by organizations such as UNICEF, WHO, and thousands of NGOs and universities) to distribute vaccines in a developing country has proven to be among ``the most successful and cost-effective health interventions'' in human history. These programs currently save between two and million deaths per year. However, even as recently as in 2008, nearly 17\% of the 8.8 million deaths in infants and young children were from diseases that could have been prevented by vaccines \cite{who:campaign_essentials}. 

Improving the effectivness of vaccine distribution programs requires improving networks in existing deployment programs as well as instituting new programs in poorly-covered countries. Our project focuses on improving an existing deployment system in Laos, which will be discussed below. Many of the processes and motivations described below are similar for other countries as well.

The current system in Laos distributes vaccines down a hierarchy on a monthly cycle. Each step lower in the hierarchy contains smaller facilities with progressively less lower capacity, fewer staff, less frequent restocking, poorer roads, and less reliable equipment, power, cell service, and Internet. Each vaccine must be stored at a specific, cool temperature to avoid spoiling, so this chain of facilities through which vaccines pass is called the ``cold chain''.

The cold chain at the lowest tier of the hierarchy operates on roughly a monthly schedule (sometimes less for particularly remote stations). Every month, workers at the next higher tier fill a small truck with ice chests full of vaccine (once at an outputs, the vaccines are kept in propane- or electicity-powered refrigerators) and then spend about two weeks driving a loop through these outputs. At each facility, the workers collect information about the previous month's demand, any necessary repairs, and the amount of vaccine spoilage. This information is used to plan the next deliveries. 

Most vaccinations are done by the Community Health Workers at these small facilities. People travel for hours by foot to get their children vaccinated, often forgoing other vital tasks. They either travel directly to the outpost, or to an event in some other community that a CHW has scheduled in advance. Sometimes CHWs pack bags with vaccines and walk around neighboring communities to do vaccinations. 

Being vaccinated is a large time investment for most individuals, so it's important that there's enough unspoiled stock to satisfy the demand for vaccinations. Individuals who travel for hours only to find that there is no vaccine left are less likely to return again, and they may recommend the same to others. 

Running out of stock is a real concern. High demand, low supply, a broken or near-broken fridge, interruptions in power supply, or temperatures high or low enough to overwhelm fridges can all cause unexpected stock outages. Giving national-, provincial-, and district-level distributors timely access to reliable information can prevent stock outages as well as mitigate the effects of outages that do occur. 

Unfortunately, the current data collection schedule means that each facility is resupplied based on data that is, on average, six weeks old. Any improvement in this aspect of distribution requires an alternate form of reporting. 

\subsection{SMS-Based Reporting}

We present SMS Immunization Manager (SIM), an easy-to-install and easy-to-customize framework that allows CHWs to report information about the vaccine cold chain (such as fridge temperature and condition) instantly by SMS. 

SIM's main features include robust SMS-based operations that correctly handle even poorly-formatted messages and a moderation web application for administrators to review the system's operation. 

The system will be deployed in Laos in the coming months and possibly to other countries after.

\section{Related Work}
What have others done - both closely and loosely related?  It is important to make sure you do a good job of this and draw upon a wide range of information sources including conferences and journals in the area, magazines and popular press, newsletters, and, of course, web searches.  Feel free to build on what was done in Winter quarter but go further.  In particular make sure you do a survey of related technologies, not just research projects.  This is where the bulk of the citations will be and should be about 1 page clearly explaining why these other works were not enough to solve the problem.

\section{Approach}
How are you going about realizing your idea?  What are the main pieces?  Why did you choose these pieces?  How do they interact?  How generally reusable will these pieces be?  This should be a fairly large section with several figures and should probably run about 1.5 to 2 pages.

\section{Implementation}
Describe the choices you had to make that were not obvious.  What was hard about getting things to work?  What tradeoffs did you have to consider?  You also want to talk about how things could have been implemented better if you had more resources or more time.  This section will also be about 1.5 to 2 pages and should include figures about the data flow and final design.

\section{Evaluation}
How are you evaluating your solution?  Evaluation has both technical aspects as well as usability aspects.  Who are your target users and who did you use as surrogates?  How well does it work?  Use both qualitative and quantitative approaches and relate the evaluation back to the scenario.  Another 1.5 to 2 page section.

\section{Conclusion and Future Work}
What are the next steps to take and why?  What lessons did you learn?  What would you differently next time?  How could the solution be improved upon further?  This section can be short and should recap the paper and back up the claims made in the abstract.  It is only about a page in length.

\section{Acknowledgements}
Who helped you along the way this quarter and how?

\bibliographystyle{abbrv}
\bibliography{report}  
\balancecolumns

\appendix
All the material you might need to reconstruct your project: source code, schematics, installation and configuration instructions, etc.  (NOTE: this does not have to conform to the template provided and can be in the most appropriate format to make it easiest to use).

\end{document}
