\documentclass{acm_proc_article-sp}

\begin{document}

\title{SMS Immunization Manager (SIM)}

\numberofauthors{4} 
\author{
       \alignauthor
       Jenny Kang\\
              \affaddr{University of Washington}\\
              \affaddr{Seattle, WA, USA}\\
              \email{jskang@uw.edu}
       \alignauthor
       Isaac Reynolds\\
              \affaddr{University of Washington}\\
              \affaddr{Seattle, WA, USA}\\
              \email{isaacr@uw.edu}
       \alignauthor 
       Jackson Roberts\\
              \affaddr{University of Washington}\\
              \affaddr{Seattle, WA, USA}\\
              \email{jcwr@uw.edu}
       \and 
       \alignauthor 
       Nicholas Shahan\\
              \affaddr{University of Washington}\\
              \affaddr{Seattle, WA, USA}\\
              \email{nshahan@uw.edu}
}

\maketitle
\begin{abstract}
Short description of your project, what you accomplished, and what conclusions 
you were able to draw. This should be in the range of 250-300 words.
\end{abstract}

\section{Introduction}
Problem description including a typical scenario of use. What problem does your project try to solve? What is interesting or hard about this problem? What do people do now?  What is your basic idea and how will it make things better? This section should include a figure or two and be approximately 2 pages.

\section{Related Work}
What have others done - both closely and loosely related?  It is important to make sure you do a good job of this and draw upon a wide range of information sources including conferences and journals in the area, magazines and popular press, newsletters, and, of course, web searches.  Feel free to build on what was done in Winter quarter but go further.  In particular make sure you do a survey of related technologies, not just research projects.  This is where the bulk of the citations will be and should be about 1 page clearly explaining why these other works were not enough to solve the problem.

\section{Approach}
How are you going about realizing your idea?  What are the main pieces?  Why did you choose these pieces?  How do they interact?  How generally reusable will these pieces be?  This should be a fairly large section with several figures and should probably run about 1.5 to 2 pages.

\section{Implementation}
Describe the choices you had to make that were not obvious.  What was hard about getting things to work?  What tradeoffs did you have to consider?  You also want to talk about how things could have been implemented better if you had more resources or more time.  This section will also be about 1.5 to 2 pages and should include figures about the data flow and final design.

\section{Evaluation}
How are you evaluating your solution?  Evaluation has both technical aspects as well as usability aspects.  Who are your target users and who did you use as surrogates?  How well does it work?  Use both qualitative and quantitative approaches and relate the evaluation back to the scenario.  Another 1.5 to 2 page section.

\section{Conclusion and Future Work}
What are the next steps to take and why?  What lessons did you learn?  What would you differently next time?  How could the solution be improved upon further?  This section can be short and should recap the paper and back up the claims made in the abstract.  It is only about a page in length.

\section{Acknowledgements}
Who helped you along the way this quarter and how?

\bibliographystyle{abbrv}
\bibliography{report}  
\balancecolumns

\appendix
All the material you might need to reconstruct your project: source code, schematics, installation and configuration instructions, etc.  (NOTE: this does not have to conform to the template provided and can be in the most appropriate format to make it easiest to use).

\end{document}
